\documentclass[10.5pt]{article}
\usepackage{amsmath,amssymb,amsthm}
\usepackage{listings}
\usepackage{graphicx}
\usepackage[shortlabels]{enumitem}
\usepackage{tikz}
\usepackage[margin=1in]{geometry}
\usepackage{fancyhdr}
\usepackage{epsfig} %% for loading postscript figures
\usepackage{amsmath}
\usepackage{float}
\usepackage{amssymb}
\usepackage{caption}
\usepackage{subfigure}
\usepackage{graphics}
\usepackage{titlesec}
\usepackage{mathrsfs}
\usepackage{amsfonts}
\usepackage{indentfirst}
\usepackage{fancybox}
\usepackage{tikz}
\usepackage{algorithm}
\usepackage{algcompatible}
\usepackage{xeCJK}
\usepackage{extarrows}
\setCJKmainfont{Kai}

\title{数学物理方法
\\第二次作业}
\author{\\董建宇   ~~2019511017}
\date{三月25日}

\begin{document}
    
\maketitle
\newpage

\section{}
证明:
\subsection{}
在三维球坐标中,利用坐标变换$x=r\sin\theta\cos\varphi,y=r\sin\theta\sin\varphi,z=r\cos\theta$,其坐标变换的Jacobe行列式为$$J=\begin{bmatrix}
    \sin\theta\cos\varphi & r\cos\theta\cos\varphi & -r\sin\theta\sin\varphi\\
    \sin\theta\sin\varphi & r\cos\theta\sin\varphi & r\sin\theta\cos\varphi\\
    \cos\theta & -r\sin\theta & 0
\end{bmatrix} = r^2\sin\theta.$$\indent
所以$\delta$函数在球坐标下的表达式为$$\delta\left(\vec{r} - \vec{r}_0\right) = \frac{1}{r^2\sin\theta}\delta(r-r_0)\delta(\theta-\theta_0)\delta(\varphi-\varphi_0) = \frac{1}{r^2}\delta(r-r_0)\delta(\cos\theta-\cos\theta_0)\delta(\varphi-\varphi_0)$$
\subsection{}
$$\nabla^2\frac{1}{\left\lvert \vec{r}-\vec{r}_0 \right\rvert} = -\nabla\cdot \frac{\vec{r}-\vec{r}_0}{\left\lvert \vec{r}-\vec{r}_0 \right\rvert ^3}$$\indent
令$\vec{l} = \vec{r}-\vec{r}_0$,则$\frac{\vec{r}-\vec{r}_0}{\left\lvert \vec{r}-\vec{r}_0 \right\rvert ^3} = \frac{\hat{l}}{l^2}$。当$\hat{l}\neq \vec{0}$时,利用求坐标可知$$\nabla\cdot\frac{\hat{l}}{l^2} = \frac{1}{l^2} \frac{\partial }{\partial l}\left(l^2 \frac{1}{l^2}\right) = 0.$$\indent
\def\ooint{{\bigcirc}\kern-11.5pt{\int}\kern-6.5pt{\int}}
当$\hat{l}=\vec{0}$时,存在$$\iiint_{R\to 0} \nabla\cdot \frac{\vec{r}-\vec{r}_0}{\left\lvert \vec{r}-\vec{r}_0 \right\rvert ^3} \,dV = \ooint \frac{\hat{l}}{l^2}\cdot\,d\vec{a} =4\pi$$\indent
所以$$\nabla^2\frac{1}{\left\lvert \vec{r}-\vec{r}_0 \right\rvert} = -\nabla\cdot \frac{\vec{r}-\vec{r}_0}{\left\lvert \vec{r}-\vec{r}_0 \right\rvert ^3} = -4\pi\delta(\vec{r}-\vec{r}_0)$$

\section{}
\subsection{}
由复数欧拉公式$e^{i\omega t} = \cos(\omega t) + i\sin(\omega t)$,可知$$\int_{-\infty}^{+\infty} \frac{\cos(\omega t)}{\omega^2 + a^2} \,d\omega = Re\left(\int_{-\infty}^{+\infty} \frac{e^{i\omega t}}{\omega^2 + a^2} \,d\omega \right) $$\indent
由留数定理可知:$$\int_{-\infty}^{+\infty} \frac{e^{i\omega t}}{\omega^2 + a^2} \,d\omega = \frac{\pi}{a} e^{-a\left\lvert t\right\rvert }$$\indent
即$$\int_{-\infty}^{+\infty} \frac{\cos(\omega t)}{\omega^2 + a^2} \,d\omega = Re\left(\int_{-\infty}^{+\infty} \frac{e^{i\omega t}}{\omega^2 + a^2} \,d\omega \right) = \frac{\pi}{a} e^{-a\left\lvert t\right\rvert }$$
\subsection{}
\subsubsection{}
由定义可知:$$F(k_1,k_2,k_3)=\sqrt{2\pi}\iiint \frac{1}{r}e^{-i2\pi\vec{r}\cdot\vec{k}} \,dx\,dy\,dz$$\indent
利用求坐标变换,其Jacobe行列式为$(J=r^2\sin\theta)$。则原积分可化为:$$\begin{aligned}
    F(k_1,k_2,k_3) &= \sqrt{2\pi}\int_0^{2\pi} \,d\varphi \int_0^{\pi} \,d\theta \int_0^{\infty} r\sin\theta e^{-i2\pi kr\cos\theta} \,dr\\
    &=-\frac{1}{\sqrt{2\pi} k^2}\int_0^{\pi} \frac{\sin\theta}{\cos^2\theta}\,d\theta = \sqrt{\frac{2}{\pi}}\frac{1}{k^2}
\end{aligned}$$\indent
\subsubsection{}
由定义可知:$$G(\vec{k})=\frac{1}{2\pi\sqrt{2\pi}}\iiint \sqrt{\frac{\pi}{2}}\frac{\delta(r-a)}{r}e^{-i \vec{r}\cdot\vec{k}}\,dx\,dy\,dz$$\indent
利用求坐标变换,其Jacobe行列式为$r^2\sin\theta$。则原积分可化为:$$\begin{aligned}
    G(\vec{k})&=\frac{1}{4\pi} \int_0^{2\pi} \,d\varphi \int_0^{\pi} \,d\theta \int_0^{\infty} \delta(r-a)r\sin\theta e^{-i kr\cos\theta}\,dr\\
    &=\frac{1}{2}\int_0^{\pi}a\sin\theta e^{-ika\cos\theta} \,d\theta = \frac{\sin ak}{k}
\end{aligned}$$

\section{}
因为f(t)的拉普拉斯变换存在,设为F(p),则有:$$F(p)=\int_0^{+\infty} f(t)e^{-pt}\,dt$$\indent
又因为f(t)为周期函数,周期为a,即f(t+a)=f(t),令u=t+a,则有:$$F(p)=\int_0^{+\infty} f(t)e^{-pt}\,dt = \int_0^{+\infty} f(t+a)e^{-pt}\,dt = e^{ap}\int_a^{+\infty}f(u)e^{-pu}\,du$$\indent
所以$$F(p)\left(1-e^{-ap}\right) = \int_0^a f(t)e^{-pt}\,dt$$\indent
即:$$F(p) = \frac{1}{1-e^{-ap}} \int_0^a f(t)e^{-pt}\,dt$$

\section{}
令$\lambda$为轻绳质量线密度,选取坐标x处一段长度为$\Delta x$微元为研究对象,由于轻绳以$\omega$角速度匀速转动,则x处对于x到l轻绳的力恰好提供圆周运动向心力,则有:$$F(x)=\int_x^l \lambda \omega^2 x\,dx=\frac{1}{2}\lambda\omega^2(l^2-x^2)$$\indent
对于研究对象,u增大为正方向,由牛顿第二定律可知:$$\lambda \Delta x\frac{\partial^2u}{\partial t^2}=-F(x)\sin(\alpha_1)-F(x+\Delta x)\sin(\alpha_2)$$\indent
由于角度很小,可以利用小角度近似有:$$\sin(\alpha_1)=\left.\frac{\partial u}{\partial x}\right\rvert_{x},~\sin(\alpha_2)=-\left.\frac{\partial u}{\partial x}\right\rvert_{x+\Delta x}$$\indent
所以得到$$\lambda \Delta x\frac{\partial^2u}{\partial t^2} = \frac{1}{2}\lambda\omega^2 \left[\left.\frac{\partial u}{\partial x}\right\rvert_{x+\Delta x}(l^2-(x+\Delta x)^2) - \left.\frac{\partial u}{\partial x}\right\rvert_{x}(l^2-x^2)\right]$$\indent
两侧同时除以$\lambda\Delta x$,并令$\Delta x\to 0$,可得振动方程为:$$\frac{\partial^2 u}{\partial t^2}=\frac{1}{2}\omega^2 \frac{\partial}{\partial x}\left[(l^2-x^2)\frac{\partial u}{\partial x}\right]$$

\section{}
由题意可知,在x=0与x=l处由恒定的热流进入,则边界条件为:$$\left.\frac{\partial u}{\partial x}\right\rvert_{x=0}=q_0,~\left.\frac{\partial u}{\partial x}\right\rvert_{x=l}=-q_0$$

\section{}
由题意可知:弦振动过程中两端为自由端,则方程及边界条件可写为:$u_x(0,t)=0,~u_x(L,t)=0$,可以得到:
$$\begin{cases}
    \frac{\partial^2u}{\partial t^2}=a^2\frac{\partial^2u}{\partial x^2}\\
    u(x,0)=\varphi(x)=\begin{cases}
        \frac{h}{c}x,0<x\leqslant c,\\
        \frac{h}{L-c}(L-x),c<x<L,
    \end{cases}\\
    u_t(x,0)=0\\
    u_x(0,t)=0,~u_x(L,t)=0
\end{cases}$$\indent

\section{}
待求问题利用叠加原理可以转化为如下两个问题:对于$(-\infty<x<+\infty,t>0)$$$\begin{cases}
    \frac{\partial^2 u}{\partial t^2} = a^2\frac{\partial^2u}{\partial x^2}\\
    u(x,0) = e^{-2x^2},u_t(x,0) = \sin(x)
\end{cases}\begin{cases}
    \frac{\partial^2 u}{\partial t^2} = a^2\frac{\partial^2u}{\partial x^2} + \cos(\omega t)\cos(x)\\
    u(x,0) = 0,u_t(x,0) = 0
\end{cases}$$\indent
对于问题一:其通解可以写成$u(x,t) = f_1(x+at)+f_2(x-at)$,带入初始边界条件可以得到:$$\left\{\begin{aligned}
    f_1(x)+f_2(x)=&e^{-2x^2}\\
    af_1'(x)-af_2'(x)=&\sin(x)
\end{aligned}\right.$$\indent
可以解得:$$\left\{\begin{aligned}
    f_1(x)=&\frac{1}{2}e^{-2x^2}+\frac{1}{2a}(1-\cos(x))+\frac{C}{2}\\
    f_2(x)=&\frac{1}{2}e^{-2x^2}-\frac{1}{2a}(1-\cos(x))-\frac{C}{2}
\end{aligned}\right.$$\indent
所以问题一的解可以写为:$$u_1(x,t)=f_1(x+at)+f_2(x-at)=\frac{1}{2}\left(e^{-2(x+at)^2}+e^{-2(x-at)^2}\right)-\frac{1}{2a}\left(\cos(x+at)-\cos(x-at)\right)$$\indent
对于问题二:利用齐次化定理,找到一个函数w(x,t;$\tau$)满足如下方程:$$\begin{cases}
    \frac{\partial^2 w}{\partial t^2} = a^2\frac{\partial^2w}{\partial x^2}, ~t>\tau>0\\
    \left. w\right\rvert _{t-\tau=0}=0,~\left. w_t\right\rvert_{t-\tau=0}=\cos(\omega t)\cos(x)
\end{cases}$$\indent
利用达朗贝尔公示,可得$$\begin{aligned}
    &w(x,t;\tau)=\frac{1}{2a}\int_{x-a(t-\tau)}^{x+a(t-\tau)} \cos(\omega t)\cos(\xi)\,d\xi\\
    &=\frac{1}{4a}\{\sin[(\omega-a)\tau+x+at]+\sin[(\omega-a)\tau-x+at]-\sin[(\omega+a)\tau-x-at]-\sin[(\omega+a)\tau+x-at]\}
\end{aligned}$$\indent
所以问题二的解为:$$\begin{aligned}
    &u_2(x,t)=\int_0^tw(x,t;\tau)\,d\tau\\ 
    &= \frac{1}{2(\omega^2-a^2)}\left[\cos(at+x)+\cos(at-x)\right]-\frac{1}{2(\omega^2-a^2)}\left[\cos(\omega t+x)+\cos(\omega t-x)\right]
\end{aligned}$$\indent
所以原问题的解为:$$\begin{aligned}
    u(x,t)=&u_1(x,t)+u_2(x,t)\\
    =&\frac{1}{2}\left(e^{-2(x+at)^2}+e^{-2(x-at)^2}\right)-\frac{1}{2a}\left(\cos(x+at)-\cos(x-at)\right)\\
    &+\frac{1}{2(\omega^2-a^2)}\left[\cos(at+x)+\cos(at-x)\right]-\frac{1}{2(\omega^2-a^2)}\left[\cos(\omega t+x)+\cos(\omega t-x)\right]
\end{aligned}$$

\section{}
由题意可知,端点x=0处不受垂直方向的力,对系统进行偶延拓得:$$\begin{cases}
    \frac{\partial^2u}{\partial t^2}=a^2\frac{\partial^2u}{\partial x^2},& -\infty<x<\infty,t>0,\\
    u(x,0)=\sin(x),u_t(x,0)=0,& 0\leqslant x<\infty,\\
    u(x,0)=\sin(-x),u_t(x,0)=0,& -\infty<x<0,\\
    u_x(0,t)=0,&t>0.
\end{cases}$$\indent
当$x-at\geqslant 0,t>0$时,由达朗贝尔公式可得:$$u(x,t) = \frac{1}{2}\left[\sin(x+at)+\sin(x-at)\right]$$\indent
当$x-at<0,t>0,x\geqslant 0$时,$-(at-x)\geqslant 0$,由达朗贝尔公式可得:$$u(x,t)=\frac{1}{2}\left[\sin(x+at)+\sin(at-x)\right]$$\indent
综上所述,初始位置为$\sin(x)$,初始速度为0的边界条件下,u的解为:$$u(x,t)=\begin{cases}
    \frac{1}{2}\left[\sin(x+at)+\sin(x-at)\right],&\text{当}x-at\geqslant 0,t>0\text{时},\\
    \frac{1}{2}\left[\sin(x+at)+\sin(at-x)\right],&\text{当}x-at< 0,t>0,x\geqslant 0\text{时}.
\end{cases}$$\indent
画图如下($0\leqslant x\leqslant 2\pi$):\\\indent
n=0: ~~~~~~~~~~~~~~~~~~~~~~~~~~~~~~~~~~~~~~~ ~~~~~~~~~~~~~~~~~~~~~~n=1:\\
\includegraphics[scale=0.4]{n=0}
\includegraphics[scale=0.4]{n=1}\\\indent
n=2: ~~~~~~~~~~~~~~~~~~~~~~~~~~~~~~~~~~~~~~~ ~~~~~~~~~~~~~~~~~~~~~~n=3:\\\\
\includegraphics[scale=0.4]{n=2}
\includegraphics[scale=0.4]{n=3}\\\indent
n=4: ~~~~~~~~~~~~~~~~~~~~~~~~~~~~~~~~~~~~~~~ ~~~~~~~~~~~~~~~~~~~~~~n=5:\\
\includegraphics[scale=0.4]{n=4}
\includegraphics[scale=0.4]{n=5}\\\indent
n=6:\\
\includegraphics[scale=0.4]{n=6}

\newpage
a
\end{document}