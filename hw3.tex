\documentclass[10.5pt]{article}
\usepackage{amsmath,amssymb,amsthm}
\usepackage{listings}
\usepackage{graphicx}
\usepackage[shortlabels]{enumitem}
\usepackage{tikz}
\usepackage[margin=1in]{geometry}
\usepackage{fancyhdr}
\usepackage{epsfig} %% for loading postscript figures
\usepackage{amsmath}
\usepackage{float}
\usepackage{amssymb}
\usepackage{caption}
\usepackage{subfigure}
\usepackage{graphics}
\usepackage{titlesec}
\usepackage{mathrsfs}
\usepackage{amsfonts}
\usepackage{indentfirst}
\usepackage{fancybox}
\usepackage{tikz}
\usepackage{algorithm}
\usepackage{algcompatible}
\usepackage{xeCJK}
\usepackage{extarrows}
\setCJKmainfont{Kai}

\title{数学物理方法
\\第三次作业}
\author{\\董建宇 ~~2019511017}
\date{4月17日}

\begin{document}
    
\maketitle
\newpage

\section{}
\subsection{}
二阶偏微分方程为$$\frac{\partial^2u}{\partial x^2}+4\frac{\partial^2u}{\partial x\partial y}+5\frac{\partial^2u}{\partial y^2}+\frac{\partial u}{\partial x}+2\frac{\partial u}{\partial y}=0$$\indent
则A=1,B=2,C=5,D=1,E=2,即$B^2-AC=-1<0$,则特征方程有两个共轭复数解。特征方程为$$\left(\frac{dy}{dx}\right)^2-4\frac{dy}{dx}+5=0$$\indent
则特征方程的解为$y=(2\pm i)x$,令$\xi =2x-y,~\eta =x$,则可得$$\begin{aligned}
    a&=\left(\frac{\partial\xi}{\partial x}\right)^2+4\frac{\partial\xi}{\partial x}\frac{\partial\xi}{\partial y}+5\left(\frac{\partial\xi}{\partial y}\right)^2=1\\
    b&=\frac{\partial\xi}{\partial x}\frac{\partial\eta}{\partial x}+2\left(\frac{\partial\xi}{\partial x}\frac{\partial\eta}{\partial y}+\frac{\partial\xi}{\partial y}\frac{\partial\eta}{\partial x}\right)+5\frac{\partial\xi}{\partial y}\frac{\partial\eta}{\partial y}=0\\
    c&=\left(\frac{\partial\eta}{\partial x}\right)^2+4\frac{\partial\eta}{\partial x}\frac{\partial\eta}{\partial y}+5\left(\frac{\partial\eta}{\partial y}\right)^2=1\\
    d&=\frac{\partial^2\xi}{\partial x^2}+4\frac{\partial^2\xi}{\partial x\partial y}+5\frac{\partial^2\xi}{\partial y^2}+\frac{\partial\xi}{\partial x}+2\frac{\partial\xi}{\partial y}=0\\
    e&=\frac{\partial^2\eta}{\partial x^2}+4\frac{\partial^2\eta}{\partial x\partial y}+5\frac{\partial^2\eta}{\partial y^2}+\frac{\partial\eta}{\partial x}+2\frac{\partial\eta}{\partial y}=1\\
    f&=0\\
    g&=0
\end{aligned}$$\indent
则该二阶偏微分方程标准形式为:$$\frac{\partial^2u}{\partial\xi^2}+\frac{\partial^2u}{\partial\eta^2}=-\frac{\partial u}{\partial\eta}$$
\subsection{}
二阶偏微分方程为$$\frac{\partial^2u}{\partial x^2}+y\frac{\partial^2u}{\partial y^2}+\frac{1}{2}\frac{\partial u}{\partial y}=0$$\indent
则A=1,B=0,C=y,D=0,E=$\frac{1}{2}$,则特征方程为$$\left(\frac{dy}{dx}\right)^2+y=0$$ \indent$\Delta=B^2-AC=-y$。\\\indent
\textcircled{1}当y>0时:特征方程的解为$2\sqrt{y}\mp ix=C$,C为常数,则令$\xi=2\sqrt{y},\eta=x$,则可得
$$\begin{aligned}
    a&=\left(\frac{\partial\xi}{\partial x}\right)^2+y\left(\frac{\partial\xi}{\partial y}\right)^2=1\\
    b&=\frac{\partial\xi}{\partial x}\frac{\partial\eta}{\partial x}+y\frac{\partial\xi}{\partial y}\frac{\partial\eta}{\partial y}=0\\
    c&=\left(\frac{\partial\eta}{\partial x}\right)^2+y\left(\frac{\partial\eta}{\partial y}\right)^2=1\\
    d&=\frac{\partial^2\xi}{\partial x^2}+y\frac{\partial^2\xi}{\partial y^2}+\frac{1}{2}\frac{\partial\xi}{\partial y}=0\\
    e&=\frac{\partial^2\eta}{\partial x^2}+y\frac{\partial^2\eta}{\partial y^2}+\frac{1}{2}\frac{\partial\eta}{\partial y}=0\\
    f&=0\\
    g&=0
\end{aligned}$$\indent
则当$y>0$时,该二阶偏微分方程标准形式为:$$\frac{\partial^2u}{\partial\xi^2}+\frac{\partial^2u}{\partial\eta^2}=0.$$\indent
\textcircled{2}当y=0时:在x轴上,u可以取任意函数,只需满足$$\frac{\partial^2u}{\partial x^2}=-\frac{1}{2}\frac{\partial u}{\partial y}.$$\indent
\textcircled{3}当y<0时:特征方程的解为$-2\sqrt{-y}\pm x=C$,C为常数,则令$\xi=-2\sqrt{-y}+x,\eta=-2\sqrt{-y}-x$,可得
$$\begin{aligned}
    a&=\left(\frac{\partial\xi}{\partial x}\right)^2+y\left(\frac{\partial\xi}{\partial y}\right)^2=0\\
    b&=\frac{\partial\xi}{\partial x}\frac{\partial\eta}{\partial x}+y\frac{\partial\xi}{\partial y}\frac{\partial\eta}{\partial y}=-2\\
    c&=\left(\frac{\partial\eta}{\partial x}\right)^2+y\left(\frac{\partial\eta}{\partial y}\right)^2=0\\
    d&=\frac{\partial^2\xi}{\partial x^2}+y\frac{\partial^2\xi}{\partial y^2}+\frac{1}{2}\frac{\partial\xi}{\partial y}=0\\
    e&=\frac{\partial^2\eta}{\partial x^2}+y\frac{\partial^2\eta}{\partial y^2}+\frac{1}{2}\frac{\partial\eta}{\partial y}=0\\
    f&=0\\
    g&=0
\end{aligned}$$\indent
则当y<0时,该二阶偏微分方程标准形式为$$\frac{\partial^2u}{\partial\xi\partial\eta}=0$$
\subsection{}
二阶偏微分方程为$$\frac{\partial^2u}{\partial x^2}-2\cos x\frac{\partial^2u}{\partial x\partial y}-(3+\sin^2x)\frac{\partial^2u}{\partial y^2}-y-\frac{\partial u}{\partial y}=0$$\indent
则A=1,B=$-\cos x$,C=$-(3+\sin^2x)$,D=0,E=-1,F=-1,G=y,则特征方程为$$\left(\frac{dy}{dx}\right)^2+2\cos x\frac{dy}{dx}-(3+\sin^2x)=0$$\indent
$\Delta=B^2-AC=4>0$,特征方程的解为$y=-\sin x\pm 2x$,令$\xi=\sin x-2x+y,\eta=\sin x+2x+y$,可得$$\begin{aligned}
    a&=\left(\frac{\partial\xi}{\partial x}\right)^2-2\cos x\frac{\partial\xi}{\partial x}\frac{\partial\xi}{\partial y}-(3+\sin^2x)\left(\frac{\partial\xi}{\partial y}\right)^2=0\\
    b&=\frac{\partial\xi}{\partial x}\frac{\partial\eta}{\partial x}-\cos x\left(\frac{\partial\xi}{\partial x}\frac{\partial\eta}{\partial y}+\frac{\partial\xi}{\partial y}\frac{\partial\eta}{\partial x}\right)-(3+\sin^2x)\frac{\partial\xi}{\partial y}\frac{\partial\eta}{\partial y}=-8\\
    c&=\left(\frac{\partial\eta}{\partial x}\right)^2-2\cos x\frac{\partial\eta}{\partial x}\frac{\partial\eta}{\partial y}-(3+\sin^2x)\left(\frac{\partial\eta}{\partial y}\right)^2=0\\
    d&=\frac{\partial^2\xi}{\partial x^2}-2\cos x\frac{\partial^2\xi}{\partial x\partial y}-(3+\sin^2x)\frac{\partial^2\xi}{\partial y^2}-\frac{\partial\xi}{\partial y}=-\sin x-1\\
    e&=\frac{\partial^2\eta}{\partial x^2}-2\cos x\frac{\partial^2\eta}{\partial x\partial y}-(3+\sin^x)\frac{\partial^2\eta}{\partial y^2}-\frac{\partial\eta}{\partial y}=-\sin x-1\\
    f&=0\\
    g&=y
\end{aligned}$$\indent
则二阶偏微分方程标准形式为$$\frac{\partial^2u}{\partial\xi\partial\eta}=-\frac{(\sin x+1)}{16}\frac{\partial u}{\partial\xi}-\frac{(\sin x+1)}{16}\frac{\partial u}{\partial\eta}+y$$

\section{}
\subsection{}
二阶微分方程为:$$a_1(x)\frac{\partial^2u}{\partial x^2}+b_1(y)\frac{\partial^2u}{\partial y^2}+a_2(x)\frac{\partial u}{\partial x}+b_2(y)\frac{\partial u}{\partial y}=0$$\indent
假设原方程有分离变量解:$$u(x,y)=X(x)Y(y)$$\indent
代入原方程,可得:$$a_1(x)X''(x)Y(y)+b_1(y)X(x)Y''(y)+a_2(x)X'(x)Y(y)+b_2(y)X(x)Y'(y)=0$$\indent
整理可得:$$a_1(x)\frac{X''(x)}{X(x)}+a_2(x)\frac{X'(x)}{X(x)}=-b_1(y)\frac{Y''(y)}{Y(y)}-b_2(y)\frac{Y'(y)}{Y(y)}$$\indent
左侧与y无关,右侧与x无关,则可令其等于一常数$-\lambda$,则有$$\begin{aligned}
    a_1(x)X''(x)+a_2(x)X'(x)+\lambda X(x)&=0\\
    -b_1(y)Y''(y)-b_2(y)Y'(y)+\lambda Y(y)&=0
\end{aligned}$$
\subsection{}
二阶偏微分方程为:$$\frac{1}{\rho}\frac{\partial}{\partial\rho}\left(\rho\frac{\partial u}{\partial \rho}\right)+\frac{1}{\rho^2}\frac{\partial^2u}{\partial\varphi^2}=0$$\indent
假设原方程有分离变量解:$$u(\rho,\varphi)=S(\rho)\Phi(\varphi)$$\indent
代入原方程,可得:$$\Phi(\varphi)\left(S''(\rho)+\frac{S'(\rho)}{\rho}\right)+\frac{S(\rho)}{\rho^2}\Phi''(\varphi)=0$$\indent
整理可得:$$\frac{\rho^2S''(\rho)+\rho S'(\rho)}{S(\rho)}=-\frac{\Phi''(\varphi)}{\Phi(\varphi)}$$\indent
左侧与$\varphi$无关,右侧与$\rho$无关,则可令其等于一常数$-\lambda$,则有$$\begin{aligned}
    \rho^2S''(\rho)+\rho S'(\rho)+\lambda S(\rho)
    &=0\\
    \Phi''(\varphi)-\lambda\Phi(\varphi)&=0
\end{aligned}$$
\subsection{}
二阶偏微分方程为:$$\frac{1}{r^2}\frac{\partial}{\partial r}\left(r^2\frac{\partial u}{\partial r}\right)+\frac{1}{r^2\sin\theta}\frac{\partial}{\partial\theta}\left(\sin\theta\frac{\partial u}{\partial\theta}\right)=0$$\indent
假设原方程有分离变量解:$$u(r,\theta)=R(r)\Theta(\theta)$$\indent
代入原方程,可得:$$\Theta(\theta)\frac{\partial}{\partial r}\left(r^2R'(r)\right)+R(r)\frac{1}{\sin\theta}\frac{\partial}{\partial\theta}\left(\sin\theta\Theta'(\theta)\right)=0$$\indent
整理可得:$$\frac{r^2R''(r)+2rR'(r)}{R(r)}=-\frac{\Theta''(\theta)+\cot\theta\Theta'(\theta)}{\Theta(\theta)}$$\indent
左侧与$\theta$无关,右侧与r无关,则可令其等于一常数$-\lambda$,则有$$\begin{aligned}
    r^2R''(r)+2rR'(r)+\lambda R(r)&=0\\
    \Theta''(\theta)+\cot\theta\Theta'(\theta)-\lambda\Theta(\theta)&=0
\end{aligned}$$

\section{}
由题意可知:方程及初始条件与边界条件$$\left\{\begin{aligned}
    &\frac{\partial^2 u}{\partial t^2}=a^2\frac{\partial^2u}{\partial x^2}(0<x<l,t>0)\\
    &\left.\frac{\partial u}{\partial x}\right\rvert_{x=0},\left.\frac{\partial u}{\partial x}\right\rvert_{x=l}=0\\
    &u\mid_{t=0}=e^{-x^2},\left.\frac{\partial u}{\partial t}\right\rvert_{t=0}=2axe^{-x^2}
\end{aligned}\right.$$\indent
设方程的解具有$u(x,t)=X(x)T(t)$的形式,则有$$\frac{T''(t)}{a^2T(t)}=\frac{X''(x)}{X(x)}=-\lambda$$\indent
其中$\lambda$为一常数。\\\indent
求解X(x):$$X''(x)+\lambda X(x)=0$$\indent
由边界条件$\left.\frac{\partial u}{\partial x}\right\rvert_{x=0},\left.\frac{\partial u}{\partial x}\right\rvert_{x=l}=0$可知,X(x)的本征值为$\lambda_n=\left(\frac{n\pi}{l}\right)^2$,则有$$X_n(x)=A_n\cos\left(\frac{n\pi}{l}x\right),~n=0,1,2,3,\dots$$\indent
求解T(t):$$T''(t)+\lambda_n a^2T(t)=0$$\indent
则$$T$$$$T_n(t)=B_n'\cos\left(\frac{n\pi a}{l}t\right)+C_n'\sin\left(\frac{n\pi a}{l}t\right),~n=1,2,3,\dots$$\indent
则有$$u_n(x,t)=X_n(x)T_n(t)=\left[B_n\cos\left(\frac{n\pi a}{l}t\right)+C_n\sin\left(\frac{n\pi a}{l}t\right)\right]\cos\left(\frac{n\pi}{l}x\right)$$\indent
线性组合出一般解为:$$u(x,t)=\sum_{n=0}^{\infty}\left[B_n\cos\left(\frac{n\pi a}{l}t\right)+C_n\sin\left(\frac{n\pi a}{l}t\right)\right]\cos\left(\frac{n\pi}{l}x\right)$$\indent
代入初始条件:$$\begin{aligned}
    B_0+\sum_{n=1}^{\infty}B_n\cos\left(\frac{n\pi}{l}x\right)&=e^{-x^2}\\
    \sum_{n=1}^{\infty}\frac{n\pi a}{l}C_n\cos\left(\frac{n\pi}{l}x\right)&=2axe^{-x^2}
\end{aligned}$$\indent
由傅立叶系数公式可得:$$\begin{aligned}
    B_0&=\frac{1}{l}\int_0^le^{-x^2}\,dx\\
    B_n&=\frac{2}{l}\int_0^le^{-x^2}\cos\left(\frac{n\pi}{l}x\right)\,dx,~n=1,2,3,\dots\\
    C_0&=\frac{a}{l}\left(1-e^{-l^2}\right)\\
    C_n&=\frac{4}{n\pi}\int_0^lxe^{-x^2}\cos\left(\frac{n\pi}{l}x\right)\,dx,~n=1,2,3,\dots
\end{aligned}$$\indent
所以原方程的解为:$$\begin{aligned}u(x,t)=
    &\frac{1}{l}\int_0^le^{-x^2}\,dx+\frac{a}{l}\left(1-e^{-l^2}\right)t\\
    &+\sum_{n=1}^{\infty}\left[\cos\left(\frac{n\pi a}{l}t\right)\frac{2}{l}\int_0^le^{-x^2}\cos\left(\frac{n\pi}{l}x\right)\,dx+\sin\left(\frac{n\pi a}{l}t\right)\frac{4}{n\pi}\int_0^lxe^{-x^2}\cos\left(\frac{n\pi}{l}x\right)\,dx\right]\cos\left(\frac{n\pi}{l}x\right)\end{aligned}$$\indent
取l=1,a=1,n加到1000.画图如下:\\\indent
t=0: ~~~~~~~~~~~~~~~~~~~~~~~~~~~~~~~~~~~~~~~ ~~~~~~~~~~~~~~~~~~~~~~t=1:\\
\includegraphics[scale=0.2]{t=0.jpg}
\includegraphics[scale=0.2]{t=1.jpg}\\\indent
t=2: ~~~~~~~~~~~~~~~~~~~~~~~~~~~~~~~~~~~~~~~ ~~~~~~~~~~~~~~~~~~~~~~t=5:\\\\
\includegraphics[scale=0.2]{t=2.jpg}
\includegraphics[scale=0.2]{t=5.jpg}\\\indent

\section{}
由题意可知:方程及初始条件与边界条件$$\left\{\begin{aligned}
    &\frac{\partial u}{\partial t}=a^2\frac{\partial^2u}{\partial x^2}(0<x<l,t>0)\\
    &u\mid_{t=0}=\frac{bx(l-x)}{l^2}\\
    &u(0,t)=u(l,t)=0
\end{aligned}\right.$$\indent
假设方程的解具有$u(x,t)=X(x)T(t)$的形式,则有$$\frac{T'(t)}{a^2T(t)}=\frac{X''(x)}{X(x)}=-\lambda$$\indent
其中$\lambda$为一常数。\\\indent
求解X(x):$$X''(x)+\lambda X(x)=0$$\indent
由边界条件$u(0,t)=u(l,t)=0$可知X(x)解的特征值为$\lambda_n=\left(\frac{n\pi}{l}\right)^2$,则有$$X_n(x)=A_n\sin\left(\frac{n\pi}{l}x\right),~n=1,2,3,\dots$$\indent
求解T(t):$$T'(t)+\lambda_n a^2T(t)=0$$\indent
则$$T_n(t)=B_nexp\left[-\left(\frac{na\pi}{l}\right)^2t\right]$$\indent
则有$$u_n(x,t)=X_n(x)T_n(t)=C_nexp\left[-\left(\frac{na\pi}{l}\right)^2t\right]\sin\left(\frac{n\pi}{l}x\right)$$\indent
线性组合出一般解为:$$u(x,t)=\sum_{n=1}^{\infty}C_nexp\left[-\left(\frac{na\pi}{l}\right)^2t\right]\sin\left(\frac{n\pi}{l}x\right)$$\indent
代入初始条件:$$u(x,0)=\frac{bx(l-x)}{l^2}=\sum_{n=1}^{\infty}C_n\sin\left(\frac{n\pi}{l}x\right)$$\indent
利用傅立叶系数公式可知$$C_n=\frac{2}{l}\int_0^l\frac{bx(l-x)}{l^2}\sin\left(\frac{n\pi}{l}x\right)\,dx=\begin{cases}
    \frac{8b}{\left(n\pi\right)^3},&n=1,3,5,\dots\\
    0,&n=2,4,6\dots
\end{cases}$$\indent
令n=2k-1,则温度分布函数为$$u(x,t)=\sum_{k=1}^{\infty}\frac{8b}{\left((2k-1)\pi\right)^3}exp\left[-\left(\frac{(2k-1)a\pi}{l}\right)^2t\right]\sin\left(\frac{(2k-1)\pi}{l}x\right)$$

\section{}
选取球坐标系,则有:$$x=r\sin\theta\cos\varphi,y=r\sin\theta\sin\varphi,z=r\cos\theta$$\indent
则波动方程可化为:$$\frac{1}{a^2}\frac{\partial^2u}{\partial t^2}=\frac{1}{r^2}\frac{\partial}{\partial r}\left(r^2\frac{\partial u}{\partial r}\right)+\frac{1}{r^2\sin\theta}\frac{\partial}{\partial \theta}\left(\sin\theta\frac{\partial u}{\partial\theta}\right)+\frac{1}{r^2\sin^2\theta}\frac{\partial^2u}{\partial\varphi^2}$$\indent
由题意可知,u不依赖于角向变量$\theta,\varphi$,则波动方程可化为:$$\frac{1}{r^2}\frac{\partial}{\partial r}\left(r^2\frac{\partial u}{\partial r}\right)=\frac{1}{a^2}\frac{\partial^2u}{\partial t^2}$$\indent
注意到:$$\frac{1}{r^2}\frac{\partial}{\partial r}\left(r^2\frac{\partial u}{\partial r}\right)=\frac{\partial^2u}{\partial r^2}+\frac{2}{r}\frac{\partial u}{\partial r}=\frac{1}{r}\frac{\partial^2(ru)}{\partial r^2}$$\indent
则有:$$\frac{\partial^2(ru)}{\partial t^2}=a^2\frac{\partial^2(ru)}{\partial r^2}$$\indent
这是一个以ru为变量的一维波动方程,其通解为$$ru(r,t)=f_1(r+at)+f_2(r-at)$$\indent
代入初始条件可得$$\left\{\begin{aligned}
    &f_1(r)+f_2(r)=ru_0,~r<R\\
    &f_1(r)+f_2(r)=0,~r>R\\
    &af_1'(r)-af_2'(r)=0
\end{aligned}\right.$$\indent
则有$$f_1(r)=\begin{cases}
    \frac{u_0}{2}r+C,&r<R\\
    C,&r>R
\end{cases}~~f_2(r)=\begin{cases}
    \frac{u_0}{2}r-C,&r<R\\
    -C,&r>R
\end{cases}$$\indent
所以振动方程的解为$$u(r,t)=\frac{f_1(r+at)+f_2(r-at)}{r}$$\indent
其中$r=\sqrt{x^2+y^2+z^2},~f_1(r+at),f_2(r-at)$由上式给出。

\section{}
选取极坐标系,则有:$$x=r\cos\theta,y=r\sin\theta$$\indent
则波动方程可化为$$\begin{cases}
    \frac{1}{r}\frac{\partial}{\partial r}\left(r\frac{\partial u}{\partial r}\right)+\frac{1}{r^2}\frac{\partial^2u}{\partial\theta^2}=6r^2(1+\sin2\theta),&0<a<r<b<\infty\\
    \left.u\right\rvert_{x^2+y^2=a^2}=1,~\left.\frac{\partial u}{\partial n}\right\rvert_{x^2+y^2=b^2}=0
\end{cases}$$\indent
设方程的解可以写成$$u(r,\theta)=\sum_{n=0}^{\infty}A_n(r)\cos n\theta+B_n(r)\sin n\theta$$\indent
则有$$\sum_{n=0}^{\infty}\left[A_{n}''(r)+\frac{1}{r}A_n'(r)-\frac{n^2}{r^2}A_n(r)\right]\cos n\theta+\left[B_{n}''(r)+\frac{1}{r}B_n'(r)-\frac{n^2}{r^2}B_n(r)\right]\sin n\theta=6r^2(1+\sin2\theta)$$\indent
所以有$$\begin{aligned}
    &A_0''(r)+\frac{1}{r}A_0'(r)=6r^2\\
    &B_2''(r)+\frac{1}{r}B_2'(r)-\frac{4}{r^2}B_2(r)=6r^2
\end{aligned}$$\indent
初始条件为$$A_0(a)=1,B_n(a)=0,A_n'(b)=0,B_n'(b)=0$$\indent
则方程的解为$$u(r,\theta)=A_0(r)+B_2(r)\sin(2\theta)$$\indent
设$$A_0(r)=c_0+d_0\ln(r)+\frac{3}{8}r^4,~B_2(r)=c_2r^2+d_2r^{-2}+r^4$$\indent
则有$$c_2a^2+d_2a^{-2}+a^4=0,~c_0+d_0+\frac{3}{8}a^4=1$$$$c_2b-d_2b^{-3}+2b^3=0,~\frac{d_0}{b}+\frac{3}{2}b^3=0$$\indent
可得$$u(r,\theta)=1-\frac{3}{8}a^4+\frac{3}{2}b^4\ln(\frac{a}{r})+\frac{3}{8}r^4+\left[-\frac{a^6+2b^6}{a^4+b^4}r^2-\frac{a^4b^4(a^2-2b^2)}{a^4+b^4}r^{-2}+r^4\right]\cos(2\theta)$$

\section{}
\subsection{}

由题意可知,振动方程为:$$\begin{cases}
    \frac{\partial^2u}{\partial t^2}=a^2\frac{\partial^2u}{\partial x^2},& -\infty<x<\infty,t>0,\\
    u(x,0)=\sin(x),u_t(x,0)=0,& 0\leqslant x<l,\\
    u_t(x,0)=0,& -l<x<0,\\
    u(l,t)=0,&t>0.
\end{cases}$$\indent
假设方程的解具有$u(x,t)=X(x)T(t)$的形式,则有$$\frac{T'(t)}{a^2T(t)}=\frac{X''(x)}{X(x)}=-\lambda$$\indent
其中$\lambda$为一常数。\\\indent
求解X(x):$$X''(x)+\lambda X(x)=0$$\indent
由边界条件$u_{x}(0,t)=0,u(l,t)=0$可知X(x)的本征值为$\lambda_n=\left[\frac{(2n+1)\pi}{2l}\right]^2$,则有$$X_n(x)=A_n\cos\left[\frac{(2n+1)\pi}{2l}x\right],~n=0,1,2,3,\dots$$\indent
求解T(t):$$T(t)''+\lambda_n a^2T(t)=0$$\indent
则:$$T_n(t)=B_n'\cos\left[\frac{(2n+1)a\pi}{2l}t\right]+C_n'\sin\left[\frac{(2n+1)a\pi}{2l}t\right]$$\indent
则一般解可以写为$$u(x,t)=\sum_{n=0}^{\infty}\left\{B_n\cos\left[\frac{(2n+1)a\pi}{2l}t\right]+C_n\sin\left[\frac{(2n+1)a\pi}{2l}t\right]\right\}\cos\left[\frac{(2n+1)\pi}{2l}x\right]$$\indent
代入初始条件可知$$\begin{aligned}
    &\sum_{n=0}^{\infty}B_n\cos\left[\frac{(2n+1)a\pi}{2l}x\right]=\sin x\\
    &\sum_{n=1}^{\infty}\frac{(2n+1)a\pi}{2l}C_n\cos\left[\frac{(2n+1)\pi}{2l}x\right]=0
\end{aligned}$$\indent
所以可得$$\begin{aligned}
    B_n&=\frac{2}{l}\int_0^l\sin x\cos\left(\frac{(2n+1)a\pi}{2l}x\right)\\
    &=\frac{4}{2l+(2n+1)a\pi}\left[1-\cos\left(\frac{2l+(2n+1)a\pi}{2l}\right)\right]+\frac{4}{2l-(2n+1)a\pi}\left[1-\cos\left(\frac{2l-(2n+1)a\pi}{2l}\right)\right]\\
    C_n&=0
\end{aligned}$$\indent
所以解为$$u(x,t)=\sum_{n=0}^{\infty}B_n\cos\left[\frac{(2n+1)a\pi}{2l}t\right]\cos\left[\frac{(2n+1)\pi}{2l}x\right]$$\indent
其中$B_n$上式给出。
当a=1,l=3$\pi$,画图如下:\\\indent
n=0: ~~~~~~~~~~~~~~~~~~~~~~~~~~~~~~~~~~~~~~~ ~~~~~~~~~~~~~~~~~~~~~~n=1:\\
\includegraphics[scale=0.3]{n=0.jpg}
\includegraphics[scale=0.3]{n=1.jpg}\\\indent
n=2: ~~~~~~~~~~~~~~~~~~~~~~~~~~~~~~~~~~~~~~~ ~~~~~~~~~~~~~~~~~~~~~~n=3:\\\\
\includegraphics[scale=0.3]{n=2.jpg}
\includegraphics[scale=0.3]{n=3.jpg}\\\indent
n=4: ~~~~~~~~~~~~~~~~~~~~~~~~~~~~~~~~~~~~~~~ ~~~~~~~~~~~~~~~~~~~~~~n=5:\\
\includegraphics[scale=0.3]{n=4.jpg}
\includegraphics[scale=0.3]{n=5.jpg}\\\indent
n=6:\\
\includegraphics[scale=0.3]{n=6.jpg}

\subsection{}
由上一问结果可知,若在该初始条件($u(x,0)=\sin(x)$)下在x=l处固定,振动的解与l无关,当l外推至无穷大时,结果近似为$$u(x,t)=\begin{cases}
    \frac{1}{2}\left[\sin(x+at)+\sin(x-at)\right],&\text{当}x-at\geqslant 0,t>0\text{时},\\
    \frac{1}{2}\left[\sin(x+at)+\sin(at-x)\right],&\text{当}x-at< 0,t>0,x\geqslant 0\text{时}.
\end{cases}$$\indent
与半无界弦振动问题一致。


\end{document}
